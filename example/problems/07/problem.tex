%% tags: [algorithm-design]

\begin{prob}
    Consider again the problem of determining whether there exists a pair of numbers
    in an array which, when added together, equal the maximum number in the array.
    Additionally, \textbf{assume that the array is sorted}.

    True or False: $\Theta(n)$ is a \textbf{tight} theoretical lower bound for this
    problem.

    \Tf{}

    \begin{soln}
        Any algorithm must take $\Omega(n)$ time in the worst case, since in the worst
        case all elements of the array must be read.

        This is tight because there is an algorithm that will compute the answer in
        worst-case $\Theta(n)$ time. Namely, this is an instance of the ``movie problem''
        from lecture, where instead of finding two numbers which add to an arbitrary
        target, we're looking for a specific target: the maximum. We saw an algorithm
        that solved the movie problem in $\Theta(n)$ time, where $n$ was the number of
        movies.
        Since the maximum
        is computed in $\Theta(1)$ time for a sorted array, we can use the same algorithm
        to solve this in $\Theta(n)$ time as well.
    \end{soln}

\end{prob}
